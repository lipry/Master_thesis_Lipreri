\section{Epigenetic Data}
In order to train and compare our supervised deep models we built a dataset of epigenetic data. We gathered the epigenetic features data from ENCODE \cite{ENCODE_data} and the enhancers and promoters labels from FANTOM \cite{FANTOM_data}. Formally we can define our data as follow: 
\begin{itemize}
    \item $L$ is the set of \emph{cell lines};
    \item $C$ is the set of \emph{chromosomes};
    \item $I$ is the set of \emph{intervals} over the natural numbers;
    \item $S_\ell\subseteq C\times I$, $\ell\in L$, is the set of \emph{sequences} (pair of chromosomes and intervals) relative to the cell line $\ell$, which can be mapped injectively to sequences of 1000; in general, $S_\ell\neq S_{\ell'}$; the mapping to the nucleotides is dependent on the pair chromosome/interval only;
    \item $Y$ is the set of \emph{labels} from FANTOM that can be associated to sequences (e.g., active promoter);
    \item for each $\ell\in L$, $F_\ell$ is a set of \emph{features} and $e_\ell:S_\ell\to\mathbf R^{F_\ell}$ is a map representing ENCODE \emph{epigenetic data};
\end{itemize}
%TODO: check cell lines description correctness 
The set of cell lines $L$ includes: \emph{A549} a lung carcinoma cell line derived from old Caucasian male, \emph{GM12878} a lymphoblastoid cell line produced from the blood of a female donor with northern and western European ancestry by EBV transformation, \emph{H1} an human embryonic stem cell line, \emph{HEK293} a human embryonic kidney cell line, \emph{HEPG2} a cell line derived from a male patient with liver carcinoma, \emph{K562} an immortalized cell line produced from a female patient with chronic myelogenous leukemia and \emph{MCF7} a mammary gland adenocarcinoma cell line. For every cell lines $\ell \in L$, the set of sequences $S_\ell$ contains the genomics locations of enhancers and promoters. $Y$ contains the labels for \emph{active enhancers} (A-E), \emph{active promoters} (A-P), \emph{inactive enhancers} (I-E), \emph{inactive promoters} (I-P).

%TODO: sistemare con riferimenti per uso media e mediana per l'imputer
Note that, for every cell lines $\ell \in L$, in vitro experiments are performed for every nucleotides of sequences. Consequently the pure ENCODE data, for every sequence $s \in S_\ell$ and epigenomic feature $f \in F_\ell$ (e.g., POLR2A), are reppresented by a $|s|$-length vectors of real numbers. Those values must be summarized in a single metrics. The most common used metrics are mean, variance and maximum, we choose to use maximum. Moreover we have a very small percentage (less than 1\%) of missing values; thus, we used as substitute the median of the others values.

\section{Feature Selection}
Explanations of the feature selection methods
\section{t-SNE Decomposition}
Explaining t-SNE decomposition results with plots.