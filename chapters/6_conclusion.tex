Within this work, we tried to understand whether multi-task deep learning methods outperform state-of-the-art models in distinguishing cis-regulatory regions across the genome. 

Our works focused on neural networks MTL. In particular, hard-parameters sharing approaches (Section~\ref{sec:MTLsection}). Classical feed-forward neural networks (single-task neural networks) was chosen as a benchmark (Section~\ref{sec:singletaskNN}). This choice is due to the excellent results in CRRs classification highlighted in previous works. To select the best hyper-parameters configurations of our models, a Gaussian Process has been used (Section~\ref{sec:gaussianprocess}).
To train our models, we gathered epigenetic features and regions labels from ENCODE and FANTOM (Section~\ref{sec:epigenomic_data}). We also introduced feature selection methods to reduce the large number of features, especially in some cell lines (Section~\ref{sec:featureselection}). Using these methods we selected two different reduced features sets. 
Analyzing the performance, after running the experimental setups described in Chapter~\ref{cap:experimental_seup}, we obtained promising results. Note that, the results obtained with both features sets are comparable and does not present significant differences. 

The single-task neural network ability to distinguish cis-regulatory regions is confirmed. In fact, we obtained excellent results in both considered metrics (Section~\ref{sec:methods_metrics}), in particular with the dataset balanced. Furthermore, our experiments highlight the ability of single-task neural networks to distinguish active promoters and active enhancers (Figure~\ref{fig:unbalanced_old_results} and Figure~\ref{fig:balanced_new_results}).

Our experiments using multi-tasks neural networks models gave promising results. In many computational tasks, as described in Section~\ref{sec:multi_results} and Section~\ref{sec:results_discussion}, they obtain better results than single-tasks neural networks.
In general, the proposed models result better in distinguishing: active enhancers and active promoters from everything else, active promoters from inactive promoters and, finally, inactive enhancer from inactive promoters. In addition, our studies have shown that, also with an MTL setting, a pyramidal structure of the network gives better results compared to fixing the size of neurons of every layer. 
Finally, we could notice that reducing the number of tasks (cell lines) involved in the training process improved the performances significantly. 

Although our works give exciting insight about multi-task learning neural networks ability to classify CRRs, there is still much work to be done. In particular, we suggest to try other MTL approaches not considered in this work: for instance neural networks based, such as soft-parameters sharing, or non-neural models. Besides, some efforts can be dedicated to studying an approach that permits to understand how much the cell lines are related. Probably, training multi-task learning models with highly related tasks lead to better performance. It is also possible that the architecture of the multi-tasks network is too complex for the problem at the hand. We could e.g. remove the output branches, reduce the number of hidden layers of each modules or both. 

The importance of recognizing regulatory elements across human DNA, due to the serious health implications in misregulation of genes, and the promising results that we obtained in our investigations makes us believe that continuing research in that direction can bring significant results. 
