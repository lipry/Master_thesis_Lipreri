Most of the human genome is composed of non-coding DNA sequences. It is estimated that just 2\% of human DNA could be considered coding DNA sequence. Non-coding DNA sections host a variety of cis-regulatory regions (CRR) that control the expression of genes. Detecting and understanding the function of these regions is an essential step for improving medical research of many diseases. 
Based on epigenomic data from Encyclopedia of DNA Elements (ENCODE) and the Functional Annotation of the Mammalian Genome (FANTOM) projects, in this work we tried to evaluate whether a deep learning approach, know as multi-task learning (MTL), can identify cis-regulatory regions genome-wide. We applied this method to a set of 165322 sequences and compared the classification results with single-task neural networks, considered the state-of-the-art for CRRs classification.
We discovered that our MTL neural networks give a higher performance in recognizing active enhancers and active promoters from everything else, active promoters from inactive promoters and inactive enhancers from inactive promoters. We found that using a pyramidal structure setting, in which every layer has fewer neurons than the previous one, is better than fixing the neurons numbers of every layer. Finally, some experiments highlight that the number of tasks involved in the training affects the results. 
This work demonstrates the potential of MTL methods in distinguishing cis-regulatory regions and inspires to continuing the development of new multi-task learning approaches. 