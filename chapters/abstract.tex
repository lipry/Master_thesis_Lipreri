Most of the human genome is composed by non-coding DNA sequences. It is estimated that just 2\% of human DNA could be considered coding DNA sequence. Non-coding DNA sections hosts a variety of cis-regulatory regions (CRR) that control the expression of genes. Detecting and understanding the function of this regions is an important step for improve medical research of many disease. 
Based on epigenomic data from Encyclopedia of DNA Elements (ENCODE) and the Functional Annotation of the Mammalian Genome (FANTOM) projects, in this work we tried to evaluating weather a novel deep learning approach, know as multi-task learning (MTL), is able to identify cis-regulatory regions genome-wide. We applied this methods to a set of 165322 sequences and compared the classification results with single-task neural network, considered the state-of-the-art for CRRs classification.
We discovered that our MTL neural networks give higher performance in recognizing active enhancers and active promoters from everything else, active promoters from inactive promoters and inactive enhancers from inactive promoters. We found that using a pyramidal structure settings, in which every layer has less neurons than the previous one, is better than fixing the neurons size. Finally, some experiment highlight that the number of tasks involved in the training affects the results. 

Despite lot of work must be done, this works demonstrate the potential of MTL methods in distinguish cis-regulatory regions and inspires to continuing the development of new multi-task learning approaches. 