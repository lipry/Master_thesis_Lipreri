\section{Classification of Regulatory Regions}
%todo: ampliare questa sezione anche con eventuali riferimenti
The aim of this work is to test whether a novel deep learning approach,
know as multi-task learning, perform better than the current state-of-art
machine learning approaches (in particular single task neural networks) in
DNA cis-regulatory regions (CRRs) classification. CRRs, such as
\emph{enhancers} and \emph{promoters}, are non-coding DNA sections which
regulate the transcription of the genes. Identifying these regions across
human DNA is a fundamental step to understand the impact of genetic
variation on phenotype. The development of new computational and
biological technology give to the researchers an impressive amount of new
data and therefore the possibility of developing new computational methods
for CRRs identification.

Two of the most important type of DNA cis-regulatory regions are promoters
and enhancers. Promoters are 100–1000 base pairs (bp) long DNA sequences
usually located at the beginning of the transcription initiation site,
they defines where the RNA polymerase should start the transcription of a
gene. Enhancers are 50 to 1500 bp DNA sequences that, when bound to
special proteins (Transcription Factors or TF), augment the transcription
of associated gene.

\section{Previous Works}
% TODO: RILEGGERE
% TODO: scrivere dove spiego supervised neural network single task e dove spiego gli algoritmi multi task learning.
With the rise of \emph{next-generation sequences} (NGS) techniques, able
to identify various aspects of genomes, we have seen the generation of an
incredible amount of biological information. These data are usually
collected by specialized consortia, among many we can mention: The ENCODE
(Encyclopedia of DNA Elements) Consortium \cite{ENCODE_data}, with the purpose of building a list of functional elements in the human genome; Roadmap Epigenomics Program \cite{ROADMAP} with the aim of create an epigenomic atlas for primary cells and tissues in human and finally FANTOM5 Project \cite{FANTOM_data} that is trying to uncover transcriptional regulatory networks based on transcript initiation positions.

This volume of NGS data has allowed the development of novel computational
methods with particular focus on both unsupervised and supervised machine
learning approaches, an extensive review of these techniques was written by Li et al. in \cite{LiMLReview}. Motivated by the small set of reliable annotations, the first works used unsupervised methods, among them very important contributions are chromHMM \cite{ernst2012chromhmm}
and Segway \cite{HoffmanSegway} that use respectively Hidden Markov Models
(HMM) and Dynamic Bayesian Network (DBN) to discover chromatin states. Due
to the low accuracy of the described unsupervised models the researcher focused their attention on using additional experimental features and
different machine learning approaches. In this contest many supervised
methods have attempted to predict enhancers regions inside DNA, for
instance Random Forest \cite{RFECS}, AdaBoost-based model
\cite{DELTA_adaboost}, multinomial logistic regression with LASSO
regularization \cite{ChenMultinomialLASSORegression}. The emergence of new
laboratory methods make possible for FANTOM5 Consortium to identify an
atlas of transcriptionally active promoters and a set of transcriptionally
active enhancers. In \cite{DEEP} the potential to distinguish enhancers by
training a support vector machine with these data is suggested.

The advancement of deep learning has made it possible to capture
high-level knowledge from data, this new approach has extremely improved
the state-of-the-art of many field such as speech recognition and computer
vision \cite{lecun2015deeplearning}. From predicting the impact of
variations on exon splicing \cite{XiongExon} to predicting proteins
secondary structure \cite{SecondaryStructure} also bioinformaticians have
applied deep learning techniques in many ways. In the context of
cis-regulatory elements classifications deep learning was used with many
beneficial effects. Deep Feature selection (DFS) was introduced to
overcome the limitations of previous feature selection models. In fact,
using a sparse regularized one-to-one linear layer on the input features
of a deep neural network, it can conveniently select a subset of features
for multiple classes and it can consider non-linearity via a deep
structure \cite{LiYifengDFS}. In order to detect the Transcription Factor
Binding Site (TFBS), Convolutional Neural Networks (CNN) \cite{LecunCNN,
MartinezCNN}, a powerful deep learning algorithm for modelling labelled
sequential data, was recently applied in DeepBind \cite{AlipanahiDeepBind}.
Another step forward in distinguish CRRs was accomplished by Wassermann
and his group in DECRES \cite{WassermannDECRES}. Basically they achieved
excellent results using FANTOM promoters and enhancers (that provide the
largest experimentally defined collection of CRRs) with the ENCODE project
genome-wide feature data (histone modifications, TF binding, RNA
transcripts, chromatin accessibility, and chromatin interactions) to
training supervised feedforward neural networks in order to detect
regulatory regions.

%- previous works for MTL methods
In this work we tried to make a step forward in predicting cis-regulatory
regions, using a new deep learning approach know as Multi-Task Learning
(MTL). Despite it has little or never been used in bioinformatics,
especially in distinguish CRRs, A lot of MTL variants has been successfully applied in many
fields, ranging from natural language processing \cite{CollobertWeston2008} and speech recognition \cite{Deng2013} to computer vision \cite{Girshick2015} and drug discovery \cite{Ramsundar2015}. An extensive overview of multi-task learning in deep neural network was written by Ruder in \cite{Ruder2017}. In Section~\ref{MTLsection} is given a detailed explanation of Hard parameter sharing neural networks, the approach used in this works.